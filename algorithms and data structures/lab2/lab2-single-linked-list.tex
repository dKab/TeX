\documentclass[12pt]{article}
\usepackage[english, russian]{babel}
%\usepackage{makecell}
%\usepackage{multirow}
%\usepackage{hhline}
\usepackage{ulem}
\usepackage{minted}
\usepackage[TS1, T2A]{fontenc}
\usepackage[utf8]{inputenc}
\usepackage{hyperref}
%\def\dontdofcolorbox{\renewcommand\fcolorbox[4][]{##4}}
%\makeatother
\usepackage[left=2cm,right=2cm, top=1cm,bottom=1.5cm,bindingoffset=0cm]{geometry}

%\usepackage{multicol}
%\usepackage{dirtree} 
%\usepackage{graphicx}
%\graphicspath{{imgs/}}
%\DeclareGraphicsExtensions{.pdf,.png,.jpg}
\begin{document}
\setlength{\parindent}{0pt}
\pagestyle{empty}
\begin{center}
\normalsize
\textbf{Федеральное государственное автономное образовательное учреждение высшего образования}

\small
\medskip 
\textbf{САНКТ-ПЕТЕРБУРГСКИЙ НАЦИОНАЛЬНЫЙ ИССЛЕДОВАТЕЛЬСКИЙ  УНИВЕРСИТЕТ ИНФОРМАЦИОННЫХ ТЕХНОЛОГИЙ, МЕХАНИКИ И ОПТИКИ}

\medskip 
\textbf{ФАКУЛЬТЕТ ПРОГРАММНОЙ ИНЖЕНЕРИИ И КОМПЬЮТЕРНОЙ ТЕХНИКИ}
\end{center}
\bigskip\bigskip\bigskip\bigskip\bigskip\bigskip\bigskip\bigskip\bigskip\bigskip\bigskip\bigskip
\begin{center}
\par\medskip\par\smallskip
\Large
 
\par\smallskip
\textbf{ОТЧЕТ} 

\textbf{ПО ЛАБОРАТОРНОЙ РАБОТЕ №2}

\large
\par\bigskip
\textbf{«Алгоритмы и структуры данных: \\ Списковые структуры данных.»}
\par\bigskip\par\bigskip\par\bigskip\par\bigskip\par\bigskip\par\bigskip
\par\bigskip\par\bigskip\par\bigskip\par\bigskip\par\bigskip\par\bigskip
\par\bigskip\par\bigskip\par\bigskip\par\bigskip\par\bigskip\par\bigskip
\end{center}
\begin{center}
\begin{tabular}{lllll}
Проверил:	 	 						& \hspace{80pt}	&	Выполнил:										&\\
Сентерев Ю.А.	 \_\_\_\_\_\_\_\_\_\_	&			    &	Студент группы P3255							&\\
«\_\_\_\_\_\_» 	 \_\_\_\_\_\_\_ 201\_г.  & 				&	Кабардинов Д. В. \_\_\_\_\_\_\_\_\_\_\_			&\\
										&				&													&\\
Оценка\hspace{12pt}\_\_\_\_\_\_\_\_\_	&				&													&\\
\end{tabular}
\par\bigskip\par\bigskip\par\bigskip
                                                  
\par\bigskip \par\bigskip
\end{center}
\par\bigskip\par\bigskip\par\bigskip\par\bigskip\par\bigskip\par\bigskip\par\bigskip\par\bigskip
\begin{center}
Санкт-Петербург
\par\bigskip
2019
\end{center}
\pagebreak
\textbf{ Цель работы:}
\begin{list}{-}{}
\item исследовать и изучить списковые структуры данных и их основные процедуры;
\item овладеть умениями и навыками написания программ по исследованию списковых структур
данных и их основных процедур на языке программирования Python;
\end{list}

\textbf{Задание} \\

5. Удалить n-ый элемент из списка.\\
15. Задача Джозефуса: n воинов из одного войска убивают каждого m-го из другого.
Требуется определить номер k начальной позиции воина, который должен будет остаться
последним. \\
\par
\textbf{Ход выполнения работы}\\

Реализация односвязного списка:
\begin{minted}{python}
class SingleLinkedList:
	def __init__(self):
		self.lst = None
	# Inserts node with info propery = value to the beginning of the list
	def unshift(self, value):
		node = ListNode(value)
		if self.lst != None:
			node.setPointerTo(self.lst)
		self.lst = node
	# Removes first node of the list
	# Returns removed node's 'info' property
	def shift(self):
		if self.lst == None:
			return None
		else:
			data = self.lst.info
			self.lst = self.lst.ptr
			return data 

	# Inserts node which info property is set to "value" param
	# after the node to which pointer p refers
	def insertAfter(self, p, value):
		node = ListNode(value)
		node.setPointerTo(p.ptr)
		p.setPointerTo(node)
	# Removes list node to which pointer of p node refers
	# Returns removed node's 'info' property
	def removeAfter(self, p):
		if p.ptr == None:
			return None
		else:
			node = p.ptr
			p.setPointerTo(node.ptr)
			return node.info
			
class ListNode:
	ptr = None

	def __init__(self, info):
		self.info = info
	
	def setPointerTo(self, node):
		self.ptr = node
\end{minted}

Функция, решающая задачу:
\begin{minted}{python}
# Removes n-th element from the list
# n is 1 based. So if it was an array, n = 1 would mean that
# elemenent at index 0 would be deleted.
# Returns info property of removed node
def removeNth(list, n):
	#find a node before the one that we need to remove, i.e. n - 1
	node = list.lst
	if n > 1:
		for i in range(n - 2):
			node = node.ptr
			if node == None:
				return
		return list.removeAfter(node)
	else: # n == 1 - removing first element
		return list.shift()
\end{minted}

Проверка правильности работы функции:
\begin{minted}{python}
# Testing:
testList = SingleLinkedList()
testList.unshift('d')
testList.unshift('c')
testList.unshift('b')
testList.unshift('a')
# list now have structure: abcd
removeNth(testList, 2) # removing "b" letter
#printing to check
letter = testList.lst
while letter.ptr != None:
	print(letter.info)
	letter = letter.ptr
print(letter.info) 
# expected result: acd
\end{minted}

\pagebreak
\textbf{Выводы}

В ходе выполнения работы мной была реализована структура данных Стек, а также функция добавляющая элемент в середину стека. Т.о. работа выполнена в полном объёме в соответствии с заданием. В результате получены навыки программирования на языке Python и изучена широко применяемая на практике структура данных - Стек.
\par\bigskip
\textbf{Список используемой литературы}:
\begin{enumerate}


\item \url{https://tinyurl.com/y3aj6pku} \\ - Linked Lists in Detail with Python Examples: Single Linked Lists

\item \url{https://docs.python.org/3.5/tutorial/index.html} - The Python Tutorial
\item Стивен Скиена - Алгоритмы. Руководство по разработке
\item Никлаус Вирт - Алгоритмы и структуры данных
\item Томас Кормен - Алгоритмы. Построение и анализ
\item \url{https://www.pythoncentral.io/singly-linked-list-insert-node/} - Python Data Structures Tutorial
\end{enumerate}
\end{document}

