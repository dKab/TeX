\documentclass[12pt]{article}
\usepackage[english, russian]{babel}
%\usepackage{makecell}
%\usepackage{multirow}
%\usepackage{hhline}
\usepackage{ulem}
\usepackage{minted}
\usepackage[TS1, T2A]{fontenc}
\usepackage[utf8]{inputenc}
\usepackage{hyperref}
%\def\dontdofcolorbox{\renewcommand\fcolorbox[4][]{##4}}
%\makeatother
\usepackage[left=2cm,right=2cm, top=1cm,bottom=1.5cm,bindingoffset=0cm]{geometry}

%\usepackage{multicol}
%\usepackage{dirtree} 
%\usepackage{graphicx}
%\graphicspath{{imgs/}}
%\DeclareGraphicsExtensions{.pdf,.png,.jpg}
\begin{document}
\setlength{\parindent}{0pt}
\pagestyle{empty}
\begin{center}
\normalsize
\textbf{Федеральное государственное автономное образовательное учреждение высшего образования}

\small
\medskip 
\textbf{САНКТ-ПЕТЕРБУРГСКИЙ НАЦИОНАЛЬНЫЙ ИССЛЕДОВАТЕЛЬСКИЙ  УНИВЕРСИТЕТ ИНФОРМАЦИОННЫХ ТЕХНОЛОГИЙ, МЕХАНИКИ И ОПТИКИ}

\medskip 
\textbf{ФАКУЛЬТЕТ ПРОГРАММНОЙ ИНЖЕНЕРИИ И КОМПЬЮТЕРНОЙ ТЕХНИКИ}
\end{center}
\bigskip\bigskip\bigskip\bigskip\bigskip\bigskip\bigskip\bigskip\bigskip\bigskip\bigskip\bigskip
\begin{center}
\par\medskip\par\smallskip
\Large
 
\par\smallskip
\textbf{ОТЧЕТ} 

\textbf{ПО ЛАБОРАТОРНОЙ РАБОТЕ №1}

\large
\par\bigskip
\textbf{«Алгоритмы и структуры данных: \\ Полустатические структуры данных. Стек.»}
\par\bigskip\par\bigskip\par\bigskip\par\bigskip\par\bigskip\par\bigskip
\par\bigskip\par\bigskip\par\bigskip\par\bigskip\par\bigskip\par\bigskip
\par\bigskip\par\bigskip\par\bigskip\par\bigskip\par\bigskip\par\bigskip
\end{center}
\begin{center}
\begin{tabular}{lllll}
Проверил:	 	 						& \hspace{80pt}	&	Выполнил:										&\\
Сентерев Ю.А.	 \_\_\_\_\_\_\_\_\_\_	&			    &	Студент группы P3255							&\\
«\_\_\_\_\_\_» 	 \_\_\_\_\_\_\_ 201\_г.  & 				&	Кабардинов Д. В. \_\_\_\_\_\_\_\_\_\_\_			&\\
										&				&													&\\
Оценка\hspace{12pt}\_\_\_\_\_\_\_\_\_	&				&													&\\
\end{tabular}
\par\bigskip\par\bigskip\par\bigskip
                                                  
\par\bigskip \par\bigskip
\end{center}
\par\bigskip\par\bigskip\par\bigskip\par\bigskip\par\bigskip\par\bigskip\par\bigskip\par\bigskip
\begin{center}
Санкт-Петербург
\par\bigskip
2019
\end{center}
\pagebreak
\textbf{ Цель работы:}
\begin{list}{-}{}
\item исследовать и изучить полустатические структуры данных (на примере стеков,
реализованных с помощью массивов);
\item овладеть навыками разработки алгоритмов и написания программ по исследованию
стеков на языке программирования Python;
\end{list}

\textbf{Задание} \\

Ввести символы, формируя из них стек.\\

5. Вставить символ ‘*’ в середину стека, если число элементов четное, или после среднего
элемента, если число элементов нечетное.\\
\par
\textbf{Ход выполнения работы}\\

Реализация стека:
\begin{minted}{python}
class Stack: 
	def __init__(self):
		self.stack = []

	def push(self, elem):
		self.stack.append(elem)

	def pop(self):
		if len(self.stack) == 0:
			return None
		else:
			return self.stack.pop()

	def empty(self):
		return len(self.stack) == 0

	def stackTop(self):
		if self.empty():
			return None
		else:
			return self.stack[len(self.stack) - 1]
\end{minted}

Создание стека и добавление в него элементов:
\begin{minted}{python}
charactersStack = Stack()

for i in range(4):
	charactersStack.push(i)

# The stack has even number of elements now.
\end{minted}
\pagebreak
Функция, добавляющая элемент в середину стека:
\begin{minted}{python}
# works only for stacks with length > 1
def insertInTheMiddle(stack, elementToInsert):
	helperStack = Stack()
	size = 0
	while stack.empty() == False:
		helperStack.push(stack.pop())
		size += 1

	if (size % 2) == 0:
		#size is even
		position = size / 2
	else:
		#size is odd
		position = (size // 2) + 1

	for i in range(size + 1):
		if i == position:
			stack.push(elementToInsert)
		else:
			stack.push(helperStack.pop())
	
	return stack
\end{minted}
Проверка результатов выполнения программы:
\begin{minted}{python}
insertInTheMiddle(charactersStack, '*')
# Checking the result:
while charactersStack.empty() == False:
	print(charactersStack.pop())
#expected result:	 3
# 			2
# 			*
# 			1
# 			0
#test the function on odd sized stack:
charactersStack2 = Stack()
for i in range(5):
	charactersStack2.push(i)

insertInTheMiddle(charactersStack2, '*')

while charactersStack2.empty() == False:
	print(charactersStack2.pop())
#expected result	  4
#			 3
# 			*
# 			2
# 			1
# 			0
\end{minted}
\pagebreak
\textbf{Выводы}

В ходе выполнения работы мной была реализована структура данных Стек, а также функция добавляющая элемент в середину стека. Т.о. работа выполнена в полном объёме в соответствии с заданием. В результате получены навыки программирования на языке Python и изучена широко применяемая на практике структура данных - Стек.
\par\bigskip
\textbf{Список используемой литературы}:
\begin{enumerate}


\item \url{https://www.pythoncentral.io/the-difference-between-a-list-and-an-array/} \\ - Python Central: Python Programming Guides and Tutorials

\item \url{https://docs.python.org/3.5/tutorial/index.html} - The Python Tutorial
\item Стивен Скиена - Алгоритмы. Руководство по разработке
\item Никлаус Вирт - Алгоритмы и структуры данных
\item Томас Кормен - Алгоритмы. Построение и анализ
\item \url{https://www.cs.cmu.edu/~adamchik/15-121/lectures/Stacks%20and%20Queues/Stacks%20and%20Queues.html} Carnegie Mellon University Web Site - Stacks and Queues
\end{enumerate}
\end{document}

