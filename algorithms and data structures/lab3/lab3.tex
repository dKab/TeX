\documentclass[12pt]{article}
\usepackage[english, russian]{babel}
%\usepackage{makecell}
%\usepackage{multirow}
%\usepackage{hhline}
\usepackage{ulem}
\usepackage{minted}
\usepackage[TS1, T2A]{fontenc}
\usepackage[utf8]{inputenc}
\usepackage{hyperref}
%\def\dontdofcolorbox{\renewcommand\fcolorbox[4][]{##4}}
%\makeatother
\usepackage[left=2cm,right=2cm, top=1cm,bottom=1.5cm,bindingoffset=0cm]{geometry}

%\usepackage{multicol}
%\usepackage{dirtree} 
%\usepackage{graphicx}
%\graphicspath{{imgs/}}
%\DeclareGraphicsExtensions{.pdf,.png,.jpg}
\begin{document}
\setlength{\parindent}{0pt}
\pagestyle{empty}
\begin{center}
\normalsize
\textbf{Федеральное государственное автономное образовательное учреждение высшего образования}

\small
\medskip 
\textbf{САНКТ-ПЕТЕРБУРГСКИЙ НАЦИОНАЛЬНЫЙ ИССЛЕДОВАТЕЛЬСКИЙ  УНИВЕРСИТЕТ ИНФОРМАЦИОННЫХ ТЕХНОЛОГИЙ, МЕХАНИКИ И ОПТИКИ}

\medskip 
\textbf{ФАКУЛЬТЕТ ПРОГРАММНОЙ ИНЖЕНЕРИИ И КОМПЬЮТЕРНОЙ ТЕХНИКИ}
\end{center}
\bigskip\bigskip\bigskip\bigskip\bigskip\bigskip\bigskip\bigskip\bigskip\bigskip\bigskip\bigskip
\begin{center}
\par\medskip\par\smallskip
\Large
 
\par\smallskip
\textbf{ОТЧЕТ} 

\textbf{ПО ЛАБОРАТОРНОЙ РАБОТЕ №3}

\large
\par\bigskip
\textbf{«Алгоритмы и структуры данных: \\ Алгоритмы поиска.»}
\par\bigskip\par\bigskip\par\bigskip\par\bigskip\par\bigskip\par\bigskip
\par\bigskip\par\bigskip\par\bigskip\par\bigskip\par\bigskip\par\bigskip
\par\bigskip\par\bigskip\par\bigskip\par\bigskip\par\bigskip\par\bigskip
\end{center}
\begin{center}
\begin{tabular}{lllll}
Проверил:	 	 						& \hspace{80pt}	&	Выполнил:										&\\
Сентерев Ю.А.	 \_\_\_\_\_\_\_\_\_\_	&			    &	Студент группы P3255							&\\
«\_\_\_\_\_\_» 	 \_\_\_\_\_\_\_ 201\_г.  & 				&	Кабардинов Д. В. \_\_\_\_\_\_\_\_\_\_\_			&\\
										&				&													&\\
Оценка\hspace{12pt}\_\_\_\_\_\_\_\_\_	&				&													&\\
\end{tabular}
\par\bigskip\par\bigskip\par\bigskip
                                                  
\par\bigskip \par\bigskip
\end{center}
\par\bigskip\par\bigskip\par\bigskip\par\bigskip\par\bigskip\par\bigskip\par\bigskip\par\bigskip
\begin{center}
Санкт-Петербург
\par\bigskip
2019
\end{center}
\pagebreak
\textbf{Задание} \\

5. Вывести на экран все числа упорядоченного массива А кратные 4 (4,8,...) с помощью
помощью линейного, бинарного и индексно-последовательного поиска. \\
\par
\textbf{ Цель работы:}
\begin{list}{-}{}
\item исследовать и изучить основные процедуры, используемые при работе с бинарными
(двоичными) деревьями;
\item овладеть умениями и навыками написания программ по исследованию бинарных деревьев
на языке программирования Python.
\end{list}


\par
\textbf{Ход выполнения работы}\\
1. Последовательный поиск\\
Реализация алгоритма последовательного поиска на языке Python:
\begin{minted}{python}
def linearSearch(numbers, search):
	for number in numbers:
		if number == search:
			return number
	return None
\end{minted}

Функция, использующая последовательный поиск и позволяющая найти все числа массива кратные 4:
\begin{minted}{python}
def findAllMatchesUsingLinearSearch(numbers, searches):
	matches = []
	for search in searches:
		match = linearSearch(numbers, search)
		if (match != None):
			matches.append(match)
	return matches
\end{minted}

Проверка результата:
\begin{minted}{python}
def testLinearSaerch(self):
		sortedNumbers = [1, 3, 5, 9, 11, 16, 24, 28, 35, 40, 89, 100]
		multiplesOfFour = [16, 24, 28, 40, 100]
		linearSearchResultList = findAllMatchesUsingLinearSearch(sortedNumbers, multiplesOfFour)
		print('The result of linear search:')
		for num in linearSearchResultList:
			print(num)
		self.assertListEqual(linearSearchResultList, multiplesOfFour)
\end{minted}
Вывод команды
\begin{minted}{bash}
dmitrii@kdv:~/projects/TeX$ python3 algorithms\ and\ data\ structures/lab3/search.py 
The result of linear search:
16
24
28
40
100
\end{minted}

2. Бинарный поиск\\
Реализация алгоритма бинарного поиска на языке Python:
\begin{minted}{python}
def binarySearch(numbers, search):
	left = 0
	right = len(numbers) - 1
	while left < len(numbers) and right >= 0:
		mid = (left + right)//2
		middleNumber = numbers[mid]
		if middleNumber < search:
			left = mid + 1
		elif middleNumber > search:
			right = mid
		else: # it's a match
			return middleNumber
	return None
\end{minted}

Функция, использующая бинарный поиск и позволяющая найти все числа массива кратные 4:
\begin{minted}{python}
def findAllMatchesUsingBinarySearch(numbers, searches):
	matches = []
	for search in searches:
		match = binarySearch(numbers, search)
		if (match != None):
			matches.append(match)
	return matches
\end{minted}
Проверка результата:
\begin{minted}{python}
def testBinarySearch(self):
		sortedNumbers = [1, 3, 5, 9, 11, 16, 24, 28, 35, 40, 89, 100]
		multiplesOfFour = [16, 24, 28, 40, 100]
		binarySearchResultList = findAllMatchesUsingBinarySearch(sortedNumbers, multiplesOfFour)
		print('The result of binary search:')
		for num in binarySearchResultList:
			print(num)
		self.assertListEqual(binarySearchResultList, multiplesOfFour)
\end{minted}

Вывод команды
\begin{minted}{bash}
dmitrii@kdv:~/projects/TeX$ python3 algorithms\ and\ data\ structures/lab3/search.py 
The result of binary search:
16
24
28
40
100
\end{minted}

3. Индексно-последовательный поиск\\
Реализация алгоритма индексно-последовательного поиска на языке Python:
\begin{minted}{python}
def linearIndexedSearch(numbers, index, search):
	start = 0
	end = len(numbers) - 1
	for entry in index:
		if entry['val'] < search:
			start = entry['ind']
		if entry['val'] > search:
			end = entry['ind']
			break
		if entry['val'] == search:
			return search
	for i in range(start, end + 1):
		if numbers[i] == search:
			return search
	return None
\end{minted}

Функция, использующая индексно-последовательный поиск и позволяющая найти все числа массива кратные 4:
\begin{minted}{python}
def findAllMatchesUsingLinearIndexedSearch(numbers, index, searches):
	matches = []
	for search in searches:
		match = linearIndexedSearch(numbers, index, search)
		if (match != None):
			matches.append(match)
	return matches
\end{minted}
Проверка результата:
\begin{minted}{python}
def testLinearIndexedSearch(self):
		index = [
			{'val': 1, 	'ind': 0}, 
			{'val': 11, 'ind': 4}, 
			{'val': 35, 'ind': 8}
		]
		sortedNumbers = [1, 3, 5, 9, 11, 16, 24, 28, 35, 40, 89, 100]
		multiplesOfFour = [16, 24, 28, 40, 100]
		linearIndexedSearchResult = findAllMatchesUsingLinearIndexedSearch(
		sortedNumbers, index, multiplesOfFour)
		print('The result of linear indexed search:')
		for num in linearIndexedSearchResult:
			print(num)
		self.assertListEqual(linearIndexedSearchResult, multiplesOfFour)
\end{minted}

Вывод команды
\begin{minted}{bash}
dmitrii@kdv:~/projects/TeX$ python3 algorithms\ and\ data\ structures/lab3/search.py 
The result of binary search:
16
24
28
40
100
\end{minted}
\textbf{Выводы}\\
Мне удалось выполнить поставленную задачу двумя 3 способами: \\
1. Последовательный поиск \\
2. Бинарный поиск \\
3. Индексно-последовательный поиск \\
Таким образом, цель работы достигнута.
\par\bigskip
\textbf{Список используемой литературы}:
\begin{enumerate}
\item \url{https://docs.python.org/3/tutorial/introduction.html#lists} \\ - Lists
\item \url{https://docs.python.org/3.5/tutorial/index.html} - The Python Tutorial
\item Стивен Скиена - Алгоритмы. Руководство по разработке
\item Никлаус Вирт - Алгоритмы и структуры данных
\item Томас Кормен - Алгоритмы. Построение и анализ
\item \url{https://www.pythoncentral.io/singly-linked-list-insert-node/} - Python Data Structures Tutorial
\end{enumerate}
\end{document}

