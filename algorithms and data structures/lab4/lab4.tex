\documentclass[12pt]{article}
\usepackage[english, russian]{babel}
%\usepackage{makecell}
%\usepackage{multirow}
%\usepackage{hhline}
\usepackage{ulem}
\usepackage{minted}
\usepackage[TS1, T2A]{fontenc}
\usepackage[utf8]{inputenc}
\usepackage{hyperref}
%\def\dontdofcolorbox{\renewcommand\fcolorbox[4][]{##4}}
%\makeatother
\usepackage[left=2cm,right=2cm, top=1cm,bottom=1.5cm,bindingoffset=0cm]{geometry}

%\usepackage{multicol}
%\usepackage{dirtree} 
%\usepackage{graphicx}
%\graphicspath{{imgs/}}
%\DeclareGraphicsExtensions{.pdf,.png,.jpg}
\begin{document}
\setlength{\parindent}{0pt}
\pagestyle{empty}
\begin{center}
\normalsize
\textbf{Федеральное государственное автономное образовательное учреждение высшего образования}

\small
\medskip 
\textbf{САНКТ-ПЕТЕРБУРГСКИЙ НАЦИОНАЛЬНЫЙ ИССЛЕДОВАТЕЛЬСКИЙ  УНИВЕРСИТЕТ ИНФОРМАЦИОННЫХ ТЕХНОЛОГИЙ, МЕХАНИКИ И ОПТИКИ}

\medskip 
\textbf{ФАКУЛЬТЕТ ПРОГРАММНОЙ ИНЖЕНЕРИИ И КОМПЬЮТЕРНОЙ ТЕХНИКИ}
\end{center}
\bigskip\bigskip\bigskip\bigskip\bigskip\bigskip\bigskip\bigskip\bigskip\bigskip\bigskip\bigskip
\begin{center}
\par\medskip\par\smallskip
\Large
 
\par\smallskip
\textbf{ОТЧЕТ} 

\textbf{ПО ЛАБОРАТОРНОЙ РАБОТЕ №4}

\large
\par\bigskip
\textbf{«Алгоритмы и структуры данных: \\ Алгоритмы сортировки.»}
\par\bigskip\par\bigskip\par\bigskip\par\bigskip\par\bigskip\par\bigskip
\par\bigskip\par\bigskip\par\bigskip\par\bigskip\par\bigskip\par\bigskip
\par\bigskip\par\bigskip\par\bigskip\par\bigskip\par\bigskip\par\bigskip
\end{center}
\begin{center}
\begin{tabular}{lllll}
Проверил:	 	 						& \hspace{80pt}	&	Выполнил:										&\\
Сентерев Ю.А.	 \_\_\_\_\_\_\_\_\_\_	&			    &	Студент группы P3255							&\\
«\_\_\_\_\_\_» 	 \_\_\_\_\_\_\_ 201\_г.  & 				&	Кабардинов Д. В. \_\_\_\_\_\_\_\_\_\_\_			&\\
										&				&													&\\
Оценка\hspace{12pt}\_\_\_\_\_\_\_\_\_	&				&													&\\
\end{tabular}
\par\bigskip\par\bigskip\par\bigskip
                                                  
\par\bigskip \par\bigskip
\end{center}
\par\bigskip\par\bigskip\par\bigskip\par\bigskip\par\bigskip\par\bigskip\par\bigskip\par\bigskip
\begin{center}
Санкт-Петербург
\par\bigskip
2019
\end{center}
\pagebreak
\textbf{Задание} \\

С использованием любого из улучшенных методов сортировки решить задачу согласно
своему варианту.
5. Составить программу для быстрой перестройки данного массива А, в котором элементы
расположены в порядке возрастания, так, чтобы после перестройки эти же элементы
оказались расположенными в порядке убывания. \\
\par
\textbf{ Цель работы:}
\begin{list}{-}{}
\item исследовать и изучить улучшенные методы сортировки на примерах метода Шелла и
быстрой сортировки;
\item овладеть навыками написания программ с использованием улучшенных методов
сортировки на языке программирования Python.
\end{list}


\par
\textbf{Ход выполнения работы}\\
1. Сортировка Шелла\\
Идея метода заключается в сравнение разделенных на группы элементов последовательности, находящихся друг от друга на некотором расстоянии. Изначально это расстояние равно d или N/2, где N — общее число элементов. На первом шаге каждая группа включает в себя два элемента расположенных друг от друга на расстоянии N/2; они сравниваются между собой, и, в случае необходимости, меняются местами. На последующих шагах также происходят проверка и обмен, но расстояние d сокращается на d/2, и количество групп, соответственно, уменьшается. Постепенно расстояние между элементами уменьшается, и на d=1 проход по массиву происходит в последний раз.
Реализация алгоритма сортировки Шелла на языке Python:
\begin{minted}{python}
def shellSort(array):
	gapSizes = [7, 3, 1]
	for gap in gapSizes:
		for startPosition in range(gap):
			gapInsertionSort(array, startPosition, gap)

def gapInsertionSort(array, low, gap):
	for i in range(low + gap, len(array), gap):
		currentvalue = array[i]
		position = i

		while position >= gap and array[position - gap] < currentvalue:
			array[position] = array[position - gap]
			position = position - gap

		array[position] = currentvalue

sortedList = [1, 2, 5, 10, 15, 43, 55, 67, 87, 90, 99, 100]
shellSort(sortedList)
print(sortedList)
\end{minted}

Результат работы программы:
\begin{minted}{bash}
dmitrii@kdv:~/projects/TeX$ python3 algorithms\ and\ data\ structures/lab4/ShellSort.py 
[100, 99, 90, 87, 67, 55, 43, 15, 10, 5, 2, 1]
\end{minted}

2. Быстрая сортировка \\

В данном примере в списке $left$ собираются элементы, меньшие q, в списке $right$ — большие $q$, а в списке $middle$ — равные $q$. Разделение на три списка, а не на два используется для того, чтобы алгоритм не зацикливался, например, в случае, когда в списке остались только равные элементы. Барьерный элемент $q$ выбирается случайным образом из списка при помощи функции choice из модуля random.

\begin{minted}{python}
import random

def quickSort(nums):
    if len(nums) <= 1:
        return nums
    else:
        q = random.choice(nums)
        left = [elem for elem in nums if elem > q]

        middle = [q] * nums.count(q)
        right = [elem for elem in nums if elem < q]
        return quickSort(left) + middle + quickSort(right)

sorted = [1, 2, 4, 6, 7, 8, 10, 13, 46, 56, 68, 98]
reversed = quickSort(sorted)
print(reversed)
\end{minted}
Результат работы программы:
\begin{minted}{bash}
dmitrii@kdv:~/projects/TeX$ python3 algorithms\ and\ data\ structures/lab4/quickSort.py 
[98, 68, 56, 46, 13, 10, 8, 7, 6, 4, 2, 1]
\end{minted}

\textbf{Выводы}\\
Мне удалось выполнить поставленную задачу двумя 2 способами: \\
1. Сортировка Шелла\\
2. Быстрая сортировка \\
Таким образом, цель работы достигнута.
\par\bigskip
\textbf{Список используемой литературы}:
\begin{enumerate}
\item \url{https://docs.python.org/3/tutorial/introduction.html#lists} \\ - Lists
\item \url{https://docs.python.org/3.5/tutorial/index.html} - The Python Tutorial
\item Стивен Скиена - Алгоритмы. Руководство по разработке
\item Никлаус Вирт - Алгоритмы и структуры данных
\item Томас Кормен - Алгоритмы. Построение и анализ
\item \url{https://www.pythoncentral.io/singly-linked-list-insert-node/} - Python Data Structures Tutorial
\end{enumerate}
\end{document}

