\documentclass[11pt]{article}
%this is for cyrillic text
\usepackage[main=russian,english]{babel}
%this is needed for changing font
\usepackage{fontspec}
\usepackage[a4paper, top=1cm]{geometry}
%without this double integral \iint doesnt work
\usepackage{mathtools}
\usepackage{physics}
%Чтобы LaTeX-овские лигатуры работали, типа тире, кавычек и прочего
\newfontfamily\cyrillicfont[Mapping=tex-text]{Times New Roman}
\usepackage{pgfplots}
\pgfplotsset{compat=1.8}
%this font has cyrillic letters
\setmainfont{Times New Roman}
\begin{document} 
\pagestyle{empty}
1).\ Поток (2ой способ для проверки – через поверхностный интеграл 1-го рода)
Часть S плоскости $\sigma$, лежащая в первом октанте, представляет собой треугольник \\

\begin{tikzpicture}
\begin{axis} [
	view = {120}{45},
	nodes near coords,
	axis lines = center,
	axis on top,
	xmin = 0, 
	xmax = 2,
	ymin = 0,
	ymax = 2,
	zmin = 0,
	zmax = 5,
	xlabel = {$x$},
	ylabel = {$y$},
	zlabel = {$z$}
  	]
\addplot3 table [x = b, y = a, z = c] {
	a      b      c        
	1      0      0
	0      1      0
	0      0      3
	1      0      0
};
\node at (axis cs:1,-0.2,0.3) {$K$};
\node at (axis cs:-0.2,1,0.3) {$L$};
\node at (axis cs:-0.2,0.15,3) {$M$};
\node at (axis cs:-0.2,0.1,0.2) {$O$};
\end{axis}
\end{tikzpicture}

Разрешив уравнение плоскости $\sigma$ относительно $z$, получим 
$$z=3 - 3x - 3y$$
Найдем частные производные этой функции
$$\frac{\partial z}{\partial x}\ =\ p(x,y)\ =\ -3$$
$$\frac{\partial z}{\partial y}=\ q(x,y)\ =\ -3$$
Радикал, стоящий в знаменателях направляющих косинусов, равен
$$\sqrt{1+p^2\left(x,y\right)+q^2\left(x,y\right)}\ =\ \sqrt{1+9+\ 9}\ =\ \sqrt{19}$$
Согласно условию задачи, $cos\gamma>\ 0$, следовательно, перед радикалом выбираем знак «+». В результате получим 
$$cos\alpha=\frac{-\left(-3\right)}{\sqrt{19}}=\frac{3}{\sqrt{19}}$$
$$cos\beta=\frac{-\left(-3\right)}{\sqrt{19}}=\frac{3}{\sqrt{19}}$$
$$cos\gamma=\frac{1}{\sqrt{19}}$$
Найдем скалярное произведение\ $\bar{a}\ \cdot\ \bar{n}$:
$$\bar{a}\cdot \bar{n}=\frac{3}{\sqrt{19}}\cdot0+\frac{3}{\sqrt{19}}\left(3y+z\right)+\left(-\frac{1}{\sqrt{19}}\right)\cdot3\left(x+y\right)=$$
$$=\frac{9y+3z-3x-3y}{\sqrt{19}}=\frac{6y+3z-3x}{\sqrt{19}}.$$
Для вычисления потока преобразуем поверхностный интеграл по части S плоскости $\sigma$ в двойной интеграл по плоской области $D_{xy}$ – проекции плоскости S на плоскость Оху. Для этого в выражении  $\bar{a}\ \cdot\ \bar{n}$ заменим $z = 3 - 3x - 3y$ и выразим дифференциал площади поверхности по формуле
$$ds=\sqrt{1+p^2\left(x,y\right)+q^2\left(x,y\right)}dxdy=\sqrt{19}dxdy$$
Получим $$Q = \iint_{D_{xy}}{(9-3y-12x)}dxdy$$
Полученное выражение представляет собой двойной интеграл по треугольнику OKL, лежащему в плоскости Oxy. Расставим пределы интегрирования и вычислим этот интеграл.
$$
Q=\iint_{D_{xy}}\left(9-3y-12\right)dxdy=\int_{0}^{1}{dx\int_{0}^{1-x}\left(9-3y-12x\right)dy}=$$
$$=\int_{0}^{1}{dx\left(9y-\frac{3y^2}{2}-12xy\right)\Bigg|_0^{1-x}}=$$
$$=\int_{0}^{1}\left(9-9x-\frac{3-6x+3x^2}{2}-12x+12x^2\right)dx=$$
$$=\int_{0}^{1}\left(9-9x-\frac{3}{2}+3x-\frac{3x^2}{2}-12x+12x^2\right)dx\ =$$
$$=\int_{0}^{1}\left(\frac{21}{2}x^2-18x+\frac{15}{2}\right)dx=\left(\frac{7}{2}x^3-9x^2+\frac{15}{2}x\right)\Bigg|_0^1=$$
$$=\frac{7}{2}-9+\frac{15}{2}=\frac{22}{2}-9=11-9=2.$$

\textbf{Ответ:} 2.

2) Циркуляция векторного поля $\bar{a}$ по контуру $l$ представляет собой криволинейный интеграл второго рода
$$C=\int_l \bar{a}\cdot d\bar{r}=\int_{l}a_xdx+a_ydy+a_zdz$$
Для нашей задачи получим $$ C = \int_{KLMK} a_xdx + a_ydy + a_zdz = \int_{KLMK} (3y + z)dy - 3(x + y)dz$$
 По условию задачи обход контура производится в направлении $K \rightarrow L \rightarrow M \rightarrow K$ \\
Применим св-во аддитивности интеграла и представим $C$ в виде суммы трёх криволинейных интегралов $I_{KL}$, $I_{LM}$, $I_{MK}$, взятых по отрезкам $KL$, $LM$, $MK$ соответственно, т.е. $$ C = I_{KL} + I_{LM} + I_{MK}$$ Найдём значения каждого из них.

a) Отрезок $KL$ представляет собой отрезок прямой, заданной системой 
\[
 \begin{cases}
    z = 0     \\
    x + y = 1
  \end{cases}
\] 
$$y = 1 - x$$ $$dy = -dx,$$
Откуда следует, что $dz=0$. При движении от точки $K$ к точке $L$ координата $x$ меняется от 1 до 0. Следовательно  \[I_{KL}=\int_1^0 -(3(1 - x))dx = -3\int_1^0 (1 - x)dx = \]
\[= -3(x - \frac{x^2}{2})\Bigg|_1^0 = -3(-1 + \frac{1}{2})=\frac{3}{2}\]
б). Отрезок $LM$ представляет собой отрезок прямой, заданной системой \[
\begin{cases}
	x = 0 \\
	y = 1- \frac{2}{3},
\end{cases}
\]
откуда следует, что $$dx = 0$$ \[dy = - \frac{1}{3}dz\]
При движении от точки $L$ к точке $M$ координата $z$ меняется от 0  до 3. Следовательно \[
I_{LM} = \int_0^3 (-(3 - z + z) \cdot \frac{1}{3} - 3(1 - \frac{z}{3}))dz = 
\]
\[= \int_0^3 (-1 - 3 + z)dz = \int_0^3 (z - 4)dz = \]
\[= (\frac{z^2}{2} -  4z)\Bigg|_0^3 = \frac{9}{2} - 4\cdot 3 = \frac{9}{2} - 12 = -\frac{15}{2} = -7.5\]

в). Отрезок $MK$ представляет собой отрезок прямой, заданной системой \[
\begin{cases}
	y = 0 \\
	x = 1 - \frac{z}{3} \text{,}
\end{cases}
\]
Откуда следует, что $dy = 0$. При движении от точки $M$ к точке $K$ координата Z меняется от 3 до 0. Следовательно 
\[I_{MK} = \int_3^0 -3(1 - \frac{z}{3})dz) = \int_3^0 (z - 3)dz =\]
\[ = (\frac{z^2}{2} - 3z)\Bigg|_3^0 = 0 - 0 - (\frac{9}{2} - 3\cdot 3) = - \frac{9}{2} + 9 = \frac{9}{2}\]
Окончательно получим \[ C = \frac{3}{2} - \frac{15}{2} + \frac{9}{2} = -\frac{3}{2} = -1.5\]
Ответ: $-1.5$
\\
\\
\end{document}

