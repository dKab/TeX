\documentclass[11pt]{article}
%this is for cyrillic text
\usepackage[main=russian,english]{babel}
%this is needed for changing font
\usepackage{fontspec}
\usepackage[a4paper, top=1cm]{geometry}
%without this double integral \iint doesnt work
\usepackage{mathtools}
\usepackage{physics}
%Чтобы LaTeX-овские лигатуры работали, типа тире, кавычек и прочего
\newfontfamily\cyrillicfont[Mapping=tex-text]{Times New Roman}
\usepackage{pgfplots}
\pgfplotsset{compat=1.8}
%this font has cyrillic letters
\setmainfont{Times New Roman}
\renewcommand{\labelenumi}{\Roman{enumi}}
\begin{document} 
\begin{center}
\begin{Large}
\textbf{Типовой расчет модуль 6 (Ряды). Вариант 6} \\
\end{Large}
\end{center}
\stepcounter{section}
\Roman{section}.
 Исследуйте сходимость числовых рядов: a) Выясните, сходится или расходится положительный ряд; б) выясните сходится или расходится знакопеременный ряд; если ряд сходится, установите характер сходимости. \\
 
 a) $\sum_{n=1}^{\infty} \frac{n^n}{2^nn!}$
\[\lim\limits_{n \to \infty} \frac{(n+1)^{n+1}}{2^{n+1}(n+1)!}\cdot \frac{2^n!}{n^n} = \frac{1}{2}\lim\limits_{n \to \infty} \frac{(n+1)^n(n+1)n!}{n!(n+1)n^n} = \frac{1}{2}\lim\limits_{n \to \infty}\frac{(n+1)^n}{n^n} = \frac{1}{2}\lim\limits_{n \to \infty}\Big(\frac{n+1}{n}\Big)^n =  \]
\[= \frac{1}{2}\lim\limits_{n \to \infty}\Big(1+ \frac{1}{n} \Big)^n = \frac{1}{2}\cdot e = \frac{e}{2} > 1\]
По признаку Даламбера ряд расходится.\\

 б) $\sum_{n=1}^{\infty} (-1)^nsin(\frac{e}{\pi})^n $
 
Данный ряд не является знакочередующимся, но если умножить его на (-1), то получим знакочередующийся ряд 
\begin{equation}
\sum_{n = 1}^\infty (-1)^{n + 1} sin (\frac{e}{\pi})^n \label{eq:1}
\end{equation}
Исходный ряд и ряд \eqref{eq:1} сходятся или расходятся одновременнон согласно \textit{свойству 1} рядов, поэтому мы можем далее исследовать ряд \eqref{eq:1}.
Последовательность абсолютных величин членов ряда $sin(\frac{e}{\pi})^n$ монотонно убывает, а общий член ряда стремится к нулю:
\[\lim\limits_{n \to \infty} (-1)^{n+1} sin(\frac{e}{\pi})^n = 0 \]
(предел произведения равен произведению пределов, предел правой части равен нулю, предел левой неопределен но ограничен сверху 1 и -1 снизу, т.о. произведение 0 и ограниченного числа есть 0).

По признаку Лейбница ряд сходится.

Чтобы установить характер сходимости, определим сходится или расходится ряд \\
\begin{equation}
\sum_{n=1}^\infty sin(\frac{e}{\pi})^n, \label{eq:2}
\end{equation} составленный из модулей членов данного ряда.
Воспользуемся для этого предельным признаком сравнения знакоположительных рядов.
Рассмотрим ряд 
\begin{equation}
\sum_{n=1}^\infty (\frac{e}{\pi})^n \label{eq:3}
\end{equation}
Ряд \eqref{eq:3} есть ряд, полученный из сходящегося ряда геометрической прогрессии ($a + aq + aq^2 + \cdots + aq^{n-1} + \cdots (a = 1, q = \frac{e}{\pi} < 1)$) путём отбрасывания первого члена. По \textit{свойству 3} рядов сходится и ряд \eqref{eq:3}.
\end{document}

