\documentclass[12pt]{article}
\usepackage[english, russian]{babel}
%\usepackage{makecell}
%\usepackage{multirow}
%\usepackage{hhline}
\usepackage{ulem}
\usepackage{minted}
\usepackage[TS1, T2A]{fontenc}
\usepackage[utf8]{inputenc}
\usepackage{hyperref}
\usepackage{dirtree}
%\def\dontdofcolorbox{\renewcommand\fcolorbox[4][]{##4}}
%\makeatother
\usepackage[left=2cm,right=2cm, top=1cm,bottom=1.5cm,bindingoffset=0cm]{geometry}

%\usepackage{multicol}
%\usepackage{dirtree} 
%\usepackage{graphicx}
%\graphicspath{{imgs/}}
%\DeclareGraphicsExtensions{.pdf,.png,.jpg}
\begin{document}
\setlength{\parindent}{0pt}
\pagestyle{empty}
\begin{center}
\normalsize
\textbf{Федеральное государственное автономное образовательное учреждение высшего образования}

\small
\medskip 
\textbf{САНКТ-ПЕТЕРБУРГСКИЙ НАЦИОНАЛЬНЫЙ ИССЛЕДОВАТЕЛЬСКИЙ  УНИВЕРСИТЕТ ИНФОРМАЦИОННЫХ ТЕХНОЛОГИЙ, МЕХАНИКИ И ОПТИКИ}

\medskip 
\textbf{ФАКУЛЬТЕТ ПРОГРАММНОЙ ИНЖЕНЕРИИ И КОМПЬЮТЕРНОЙ ТЕХНИКИ}
\end{center}
\bigskip\bigskip\bigskip\bigskip\bigskip\bigskip\bigskip\bigskip\bigskip\bigskip\bigskip\bigskip
\begin{center}
\par\medskip\par\smallskip
\Large
 
\par\smallskip
\textbf{ОТЧЕТ} 

\textbf{ПО ЛАБОРАТОРНОЙ РАБОТЕ №1}

\large
\par\bigskip
\textbf{«Языки системного программирования: \\ Работа с файловой системой в языке высокого уровня Python»}
\par\bigskip\par\bigskip\par\bigskip\par\bigskip\par\bigskip\par\bigskip
\par\bigskip\par\bigskip\par\bigskip\par\bigskip\par\bigskip\par\bigskip
\par\bigskip\par\bigskip\par\bigskip\par\bigskip\par\bigskip\par\bigskip
\end{center}
\begin{center}
\begin{tabular}{lllll}
Проверил:	 	 						& \hspace{80pt}	&	Выполнил:										&\\
Сентерев Ю.А.	 \_\_\_\_\_\_\_\_\_\_	&			    &	Студент группы P3255							&\\
«\_\_\_\_\_\_» 	 \_\_\_\_\_\_\_ 201\_г.  & 				&	Кабардинов Д. В. \_\_\_\_\_\_\_\_\_\_\_			&\\
										&				&													&\\
Оценка\hspace{12pt}\_\_\_\_\_\_\_\_\_	&				&													&\\
\end{tabular}
\par\bigskip\par\bigskip\par\bigskip
                                                  
\par\bigskip \par\bigskip
\end{center}
\par\bigskip\par\bigskip\par\bigskip\par\bigskip\par\bigskip\par\bigskip\par\bigskip\par\bigskip
\begin{center}
Санкт-Петербург
\par\bigskip
2019
\end{center}
\pagebreak
\textbf{Задание:} \\
Составить программу которая создает резервные копии важных файлов \\
\par

\textbf{ Цель работы:}
\begin{list}{-}{}
\item Рассмотреть какими возможностями для работы с файловой системой обладает язык Python на примере модулей os и pathlib
\item На основании полученных знаний попытаться ответить на вопрос о корректности отнесения языка Python к языкам системного программирования
\end{list}

\textbf{ Задачи:}
\begin{enumerate}
\item Составить программу, создающую резервные копии файлов с использованим модуля $pathlib$
\item Составить программу, с тем же функционалом, что в задаче 1, но используя модуль $os$
\end{enumerate}

\textbf{Ход выполнения работы}\\

1. Решим поставленную задачу с использованием модуля $pathlib$:
\begin{minted}{python}
from argparse import ArgumentParser
from pathlib import Path
from datetime import datetime
import tarfile

parser = ArgumentParser()
parser.add_argument('src', help='''Source directory. \
	If it is not specified, current directory \
	will be used as source.''', nargs='?', default='.')
parser.add_argument('dest', help='''destination directory -\
	 all important files from source directory will be placed here''',
	 nargs='?', default=None)
args = parser.parse_args()
srcPath = Path(args.src)
# get list of important files
files = [x for x in srcPath.glob('**/important*') if not x.is_dir()]
# create archive in the source directory
archiveName = datetime.now().strftime('%Y-%m-%d') + '.tar.gz'
archivePath = str(srcPath.joinpath(archiveName))
tar = tarfile.open(archivePath, 'w:gz')
for file in files:
	tar.add(str(file), file.name, False)
tar.close()
# create destination directory if it doesn't exist already
dest = args.dest
backupDir = Path(dest) if dest else Path.home() / 'backup'
if not backupDir.exists():
	backupDir.mkdir(0o777, parents=True)
destPath = backupDir / archiveName
# move archive to destination directory
archive = Path(archivePath)
archive.rename(destPath)
\end{minted}
\pagebreak
Программа, код которой приведен выше, работает следующим образом. Из аргументов командой строки считываются путь директории, из которой необходимо произвести резервное копирование файлов, а также путь, по которому будет сохранён архив со всеми важными файлами, найденными в этой директории и всех её поддиректориях. Программа выбирает файлы, которые необходимо скопировать, инспектируя названия файлов: имена файлов, подлежащих резервному копированию должны иметь префикс $important$. Например, имея следующую структуру директорий: \\
\begin{quote}
\begin{minipage}{\linewidth}
\dirtree{%
    .1 test.
    .2 1.txt.
    .2 important.
    .3 important\_something.py.
    .2 important\_document.txt.
}
\end{minipage}
\end{quote}
и передав программе путь к директории tes в качестве первого аргумента, резервные копии будут созданы для файлов $important\_something.py$ и $important\_document.txt$. Эти файлы будут упакованы в архив tar и сжаты с помощью gzip. Название архива будет сгенерировано из текущей даты (например 2019-04-13.tar.gz). Если при вызове программы вторым аргументом командной строки будет передан путь, архив будет сохранён по этому пути. Если же второй аргумент опущен, в домашней директории пользователя будет создана папка $backup$ и архив будет сохранён в неё. Если же программа вызвана без аргументов, в качестве директории-источника будет использована текущая директория.\\

Таким образом задача 1 решена. Модуль pathlib предоставляет удобный интерфейс для работы с файловой системой.\\

2. Перепишем программу, на этот раз используя модуль os
\begin{minted}{python}
from argparse import ArgumentParser
from datetime import datetime
import tarfile
import os
from glob import glob
parser = ArgumentParser()
parser.add_argument('src', help='''Source directory. If it is not \
	specified current directory \
	will be used as source.''', nargs='?', default=os.getcwd())
parser.add_argument('dest', help='''destination directory -\
	 all important files from source directory will be placed here''',
	 nargs='?', default=None)
args = parser.parse_args()
srcPath = args.src
# get list of important files
files = glob(srcPath + '/**/important*', recursive=True)
files = filter(lambda file: os.path.isfile(file), files)
# create archive in the source directory
archiveName = datetime.now().strftime('%Y-%m-%d') + '.tar.gz'
archivePath = str(os.path.join(archiveName))
tar = tarfile.open(archivePath, 'w:gz')
for file in files:
	tar.add(file, os.path.basename(file), False)
tar.close()
# create destination directory if it doesn't exist already
dest = args.dest
backupDir = dest if dest else os.path.join(os.path.expanduser('~'), 'backup')
if not os.path.exists(backupDir):
	os.makedirs(backupDir)
destPath = os.path.join(backupDir, archiveName)
# move archive to destination directory
os.rename(archivePath, destPath)
\end{minted}
Программа работает идентично исходному варианту, и не слишком отличается от него по структуре. Как видно, оба модуля хорошо справляются с задачей и предоставляют все необходимые функции, которые могут потребоваться при работе с файлами, как то: запись, чтение, удаление, копирование и перемещение файлов. Следует также ометить, что работа с файловой системой является вообще говоря платформо-зависимой - В Windows есть ряд ограничений, накладываемых на названия файлов. В POSIX-совместимых системах разделителями фрагментов пути является прямой слэш (/), а в Windows - обратный, и тд. При этом оба рассмотренных модуля предоставляют интерфейс, абстрагирующий данную специфику в разлиных операционных системах, что конечно же оказывается весьма удобным. \\

\textbf{Выводы:}\\
Нам удалось выполнить поставленную задачу двумя способами.
В результате можно сделать вывод, что язык программирования Python предоставляет все возможности для полноценной работы с файлами, и это одна из причин почему его можно отнести к языкам системного программирования.
Таким образом, ответ на поставленный вопрос получен, цель работы достигнута.

\par\bigskip
\textbf{Список используемой литературы}:
\begin{enumerate}
\item \url{https://tinyurl.com/y3aj6pku} \\ - Linked Lists in Detail with Python Examples: Single Linked Lists
\item \url{https://docs.python.org/3.5/tutorial/index.html} - The Python Tutorial
\item \url{https://docs.python.org/3/library/pathlib.html} - pathlib documentation
\item \url{https://docs.python.org/3/library/os.html#module-os} - os documentation
\item \url{https://docs.python.org/3/library/argparse.html} - argparse documentation
\item \url{https://docs.python.org/3/library/tarfile.html#module-tarfile} - \\\ tarfile documentation
\end{enumerate}
\end{document}

